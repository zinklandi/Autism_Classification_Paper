\subsection{Evaluation der Algorithmen} \label{sec:evaluation}
Die Ergebnisse der angewandten Algorithmen zeigen für die Problemstellung stets eine gute Trennschärfe (siehe Tabelle \ref{tbl:results_table_without} und \ref{tbl:results_table}).

\begin{table}[htbp]
\caption{Die Resultate der angewandten Algorithmen ohne das Merkmal \glqq result\grqq}
\label{tbl:results_table_without}
\begin{tabular}{l c c c}
\textbf{Algorithmus} & \textbf{ACC} & \textbf{TPR} & \textbf{FPR} \\ \hline
Decision Tree & 92.2\% & 85.0\% & 14.9\% \\
lineare SVM & 95.0\% & 95.1\% & 4.8\% \\
RBF SVM & 97.8\% & 95.8\% & 4.1\% \\
K Nearest Neighbour & 96.9\% & 93.0\% & 6.9\%\\ 
K Means & 92.9\% & 78.2\% & 21.8\%\\ 
\end{tabular}
\centering
\end{table}

\begin{table}[htbp]
\caption{Die Resultate der angewandten Algorithmen inkl. dem Merkmal \glqq result\grqq}
\label{tbl:results_table}
\begin{tabular}{l c c c}
\textbf{Algorithmus} & \textbf{ACC} & \textbf{TPR} & \textbf{FPR} \\ \hline
Decision Tree & 100\% & 100\% & 0\% \\
lineare SVM & 98.5\% & 98.2\% & 1.7\% \\
RBF SVM & 98.1\% & 100\% & 0\% \\
K Nearest Neighbour & 99.1\% & 96.7\% & 3.0\%\\ 
K Means & 90.8\% & 77.1\% & 22.9\%\\ 
\end{tabular}
\centering
\end{table}

Ein Vergleich der Algorithmen zeigt hierbei, dass anhand des Merkmals \textit{result} stets eine bessere Trennschärfe erreicht wird. In der Datenanalyse in Kapitel \ref{sec:analysis} konnte diesbezüglich bereits gezeigt werden, dass anhand des Merkmals \textit{result} die Problemstellung eindeutig gelöst werden kann. Dennoch ist zu bemerken, dass ein Weglassen des Merkmals kaum zu einer Verschlechterung der Algorithmen führt. Für die Anwendung in der Diagnostik bieten sich vor allem der Algorithmus K-Nearest-Neighbour, sowie die Support Vector Machine zur Verhaltensanalyse an. Diese liefern im Vergleich die beste Trennschärfe unabhängig von der Verwendung des Merkmals \textit{result}. Ist die Klassifikation jedoch wie durch \textsc{Nice} \cite{NICE2012}, mit einer Verwendung des Attributes \textit{result}, definiert und durchzuführen, so ist zu einer Verwendung des Decision Tree zu raten, da dieser durch die vereinfachte Klassifikation die höchste Trennschärfe bietet.