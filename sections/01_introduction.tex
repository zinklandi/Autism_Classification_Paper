\section{Einführung}
Die Zahl der Diagnosen von Autismus-Spektrum-Störungen (ASS) steigt nach \textsc{Weintraub} \cite{Weintraub2011} jährlich stetig an. Diese Diagnosen basieren dabei auf Diagnosekriterien aus einem Katalog von Verhaltens- und Interessensmustern \cite{Weintraub2011, Thabtah2017, Thabtah2018, VanElst2014}. Bei der Diagnose einer ASS werden statistische Diagnosekriterien, basierend auf dem DSM$^{\ref{foot:1}}$-IV, DSM$^{\ref{foot:1}}$-5 und ICD\footnote{\label{foot:2}Abkürzung für \glqq International Statistical Classification of Diseases and Related Health Problems\grqq{}.}-10, zur Detektion verwendet \cite{Thabtah2017, VanElst2014}. Die Bewertung dieser Diagnosekriterien unterliegt dabei dem behandelnden Facharzt und hängt somit auch von dessen persönlicher Einschätzung und Erfahrung ab. Der Ablauf einer Diagnose entspricht dabei einer, aus dem Bereich des maschinellen Lernens und der Biometrie bekannten Vorgehensweise zur Klassifikation von Verhaltensmustern und bietet somit eine Möglichkeit zur Anwendung von Algorithmen aus diesem Bereich an.

\section{Problembeschreibung}
Der für dieses Projekt vorliegende Datensatz wurde bereits von \textsc{Thabtah} \cite{Thabtah2017, Thabtah2018} zur Erstellung eines Konzepts für maschinelle Lernverfahren für die Autismus-Diagnostik verwendet.
% Gibt es Möglichkeiten zur Verbesserung der Probleme?
In diesem Projekt sollen nun konkrete Algorithmen, anhand der Erkenntnisse von \textsc{Thabtah}, zur hinweisenden Diagnose einer ASS verglichen werden. %Dabei können die neu gewonnenen Resultate über die Veröffentlichung als Open Source für neue Entwicklungen zur Unterstützung von Ärzten in der Diagnostik einer ASS verwendet werden. 
Der Vergleich der Algorithmen erfolgt dabei über die aus der Biometrie und des maschinellen Lernens bekannten Verfahren zur Analyse von Verhaltensmustern anhand statistischer Verhaltensanalysen.