\section{Merkmalsextraktion}
Basierend auf der Beschreibung des Open Source Datensatzes und der erfolgten Datenanalyse können zunächst die in Tabelle \ref{tbl:features} dargestellten Merkmale extrahiert werden. Dabei wird in dieser Arbeit ein Merkmalsvektor $\vec{v}_M$ ermittelt.

\begin{table}
\caption{Index Beschreibung des Merkmalsvektors}
\begin{tabular}{c p{5cm}}
\textbf{Index} & \textbf{Attribut} \\ \hline
0-9 & A1 - A10 (Antworten von Frage 1 bis 10)\\
10 & result\\
11 & age \\
12 & gender\\
13 & jaundice\\
14 & autism\\
15 & relation self\\
16 & relation parent\\
17 & relation healthcare\\
18 & relation relative\\
19 & relation others\\
\end{tabular}
\centering
\label{tbl:features}
\end{table}

Somit ergibt sich aus der Tabelle \ref{tbl:features} ein Merkmalsvektor $\vec{m} = (m_0, ..., m_n)^T$ mit $n = 19$ und insgesamt 20 Merkmalen. Es ist jedoch zu beachten, dass aufgrund der Datenanalyse, das Merkmal \textit{result} die Problemstellung für den Datensatz bereits löst. Aus diesem Grund werden in der Merkmalsextraktion zwei Vektoren generiert. Dies ist der Merkmalsvektor $\vec{m}_{\text{with-result}}$, welcher das Merkmal \textit{result} enthält, sowie ein Merkmalsvektor $\vec{m}_{\text{without-result}}$ der das Merkmal nicht enthält. Dies ermöglicht in der Evaluation einen Vergleich, ob eine Klassifikation ohne das Merkmal \textit{result} mit gleicher Qualität durchgeführt werden kann.
