\section{Merkmalsextraktion}
Basierend auf der Beschreibung des Open Source Datensatzes und der erfolgten Datenanalyse können zunächst folgende Merkmale extrahiert werden. Dabei wird ein Merkmalsvektor $\vec{v}_M$ ermittelt. Dieser enthält folgende Merkmale:

\begin{itemize}
	\item \textbf{Index 0-9:} A1 - A10 (Antworten von Frage 1 bis 10)
	\item \textbf{Index 10:} result
	\item \textbf{Index 11:} age 
	\item \textbf{Index 12:} gender
	\item \textbf{Index 13:} jaundice
	\item \textbf{Index 14:} autism
	\item \textbf{Index 15:} relation self
	\item \textbf{Index 16:} relation parent
	\item \textbf{Index 17:} relation healthcare
	\item \textbf{Index 18:} relation relative
	\item \textbf{Index 19:} relation others
\end{itemize}

Somit ergibt sich durch die Merkmalsextraktion der Vektor $\vec{m} = (m_1, ..., m_n)^T$ mit $n = 20$ Merkmalen. Es ist jedoch zu Beachten, dass aufgrund der Datenanalyse das Merkmal \textit{result} die Problemstellung für den Datensatz bereits löst. Aus diesem Grund werden in der Merkmalsextraktion zwei Vektoren generiert. Dies der Merkmalsvektoren $\vec{m}_{\text{with-result}}$, welcher das Merkmal \textit{result} enthält, sowie ein Merkmalsvektor $\vec{m}_{\text{without-result}}$ der das Merkmal nicht enthält. Dies ermöglicht einen Vergleich, ob eine Diagnose ohne das Merkmal \textit{result} mit gleicher Qualität durchgeführt werden kann.
