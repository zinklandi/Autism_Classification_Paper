\subsection{K-Means}
Der Algorithmus K-Means zählt im Gegensatz zu den anderen Algorithmen, im Rahmen dieser Arbeit zu den Algorithmen des unüberwachten Lernens und wird in der Regel zum Gruppieren (engl. \glqq Clustering\grqq) von Daten verwendet. Dabei ermittelt der Algorithmus einen Fixpunkt von einer Menge von $k$ Vektoren. Dieser Fixpunkt entspricht dabei dem Mittelwertvektor $\vec{k}_m$ der trainierten Vektoren und definiert das Zentrum eines Clusters.
Die Evaluation und Klassifikation geschieht hierbei, wie bereits in Kapitel \ref{sec:kneighbour} durchgeführt, mit Hilfe einer Kreuzvalidierung. Hierzu wird ebenso die Anzahl der $k$ Merkmalsvektoren zur Berechnung des Mittelwertvektors $\vec{k}_m$ variiert. 
Zur Bewertung der Ergebnisse müssen jedoch die resultierenden Cluster-Zuordnungen des implementierten Algorithmus, unter Verwendung des \textit{sklearn} Frameworks, analysiert werden. Dabei wird innerhalb eines Clusters die Anzahl der positiv und negativ diagnostizierten Datensätze verglichen. Das Cluster wird im Anschluss der Klasse mit der größten Menge von Datensätzen einer Klasse zugeordnet.
Das Resultat des Algorithmus zeigt dabei eine gute Trennschärfe von ca. 90\%, welche jedoch im Rahmen dieser Arbeit der geringsten Genauigkeit entspricht.