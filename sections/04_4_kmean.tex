\subsection{K-Means}
Der Algorithmus K-Means zählt im Gegensatz zu den anderen Algorithmen innerhalb dieser Arbeit zu den Algorithmen des unüberwachten Lernens und wird in der Regel zum Gruppieren (engl. \glqq Clustering\grqq) von Daten verwendet. Dabei ermittelt der Algorithmus einen Fixpunkt von einer Menge von $k$ Vektoren. Dieser Fixpunkt entspricht dabei dem Mittelwertvektor $\vec{k}_m$ der trainierten Vektoren und definiert das Zentrum eines Clusters.
Die Evaluation und Klassifikation geschieht hierbei wie bereits in Abschnitt \ref{sec:kneighbour} beschrieben mit Hilfe einer Kreuzvalidierung. Dabei werden die Anzahl der $k$ Merkmalvektoren zur Berechnung des Mittelwertvektors $\vec{k_m}$ variiert. 
Zur Bewertung der Ergebnisse müssen jedoch die resultierenden Cluster-Zuordnungen des implementierten Algorithmus, unter Verwendung des \textit{sklearn} Frameworks, analysiert werden. Dabei werden innerhalb eines Clusters die Anzahl der positiv und negativ diagnostizieren Datensätze verglichen. Das Cluster wird im Anschluss der Klasse mit der größten Anzahl von Datensätzen innerhalb des Clusters zugeordnet.
Das Resultat des Algorithmus zeigt dabei gute Trennschärfe von ca. 90\%, welche jedoch innerhalb dieser Arbeit die geringste Genauigkeit bedeutet.