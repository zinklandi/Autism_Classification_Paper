\section{Fazit und Ausblick}
Anhand der Ergebnisse aus Kapitel \ref{sec:algorithms} und \ref{sec:evaluation} lässt sich die Möglichkeit des Einsatzes von maschinellen Lernmethoden zur Diagnose von ASS bestätigen. Es können dabei durch die Anwendung eines geeigneten Algorithmus sehr gute Ergebnisse zur Tendenz einer Diagnose ermittelt werden. Diese können dabei dem behandelnden Arzt eine objektive Sichtweise der Bewertung des Verhaltens ermöglichen. Jedoch ist bei einem Einsatz im Bereich der Diagnose ein Ergebnis der Methoden, aufgrund der hohen Falschdiagnosen von bis zu 7\%, nur als Hinweis zu bewerten. 
Der verwendete Datensatz von \textsc{Thabtah} \cite{Thabtah2017, Thabtah, Thabtah2018} enthält zudem bereits normierte Antworten. Somit kann die Gewichtung der Antwort zur Analyse des Verhaltens nicht mit einbezogen werden. Ebenso enthalten die Datensätze eine Klassifizierung nach dem Vorgaben von \textsc{NICE} \cite{NICE2012}, welche die Problemstellung bereits vereinfacht durch die Gesamtpunktzahl der Antworten löst (siehe Abschnitt \ref{sec:analysis} und \ref{sec:tree}). Abschließend lässt sich dennoch bestätigen, dass der Einsatz von Methoden des maschinellen Lernens in der Diagnose diese auch positiv unterstützen kann.
